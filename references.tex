\documentclass{article}

\begin{document}


\title{A USER FRIENDLY LUNG SIMULATION FOR THE INSPIRED SINEWAVE DEVICE TO MEASURE LUNG FUNCTION AND PULMONARY BLOOD FLOW 
CONCEPT PAPER}

\maketitle



\section{INTRODUCTION}
Lung simulation in medicine is one of the most effective ways for students to acquire practical knowledge.
It is also helpful for a hospital’s medical staff to practice certain clinical scenarios if needed, before facing them in a real field environment. The current project is aimed on creating a user friendly lung simulation for the inspired sinewave device to measure lung function and pulmonary blood flow.
This can be accomplished with a sinewave device (spirometer).Spirometry is the most frequently used measure of lung function and is a measure of volume against time. The spirometer is an instrument that measures the amount of air breathed in and/or out and how
 quickly the air is inhaled and exhaled from the lungs while breathing through a mouthpiece. The measurements are recorded on a device called a spirograph. The main purpose of the simulation is to offer the students the possibility of interpreting the graphs and signals that the ventilator is emitting, while it is connected to a patient.The need of a new lung simulator comes with the high prices of the advanced types already available on the market and with the limited possibilities offered by the cheaper versions.
The prototype designed during the project offers a cheap solution that includes most of the features needed to ensure the realism of the simulation. The prototype was designed using procedures already known for the devlopment of the lung simulators, in a very innovative way. The prototype is designed to improve the actual training methods used by the hospitals.

\section{OBJECTIVES}
The aim of the project is to design a reliable, economical and portable device which will be used for educational and research purposes and also lung simulation. The principal objectives that had to be achieved within the project are;
1. Increase the knowledge acquired by students while using a reliable lung simulator.
2. An eco-friendly product
3. Deliver a compelling and user experience that raises the customers approval ratings

\section{PROBLEM STATEMENT}
The existing lung simulators are very expensive and sophisticated to be incorporated in a simulation room for educational needs. While the simplest ones does not meet some of the hospital requirements and do not have a good sensitivity calibration of parameters which is needed in order to teach different clinical situations of the patients.
The main idea of the project is to create a lung simulator which is able to operate passively, physically and mechanically simulating the real lung functions, not from a computerized mathematical model of a lung simulator. It must be able to connect to the air source which will be introduced from the hospital air source ventilation system and manipulate the air through the changing of the parameters, simulating the real function of the lungs in various clinical scenarios.

\section{METHODOLOGY}
This involved developing an abstract prototype which is evaluated in weekly basis to determine its capability to meet the intended purpose. With this method, we shall be able to add features into the prototype as we shall discover. We shall also remove any unneccessary modules to increase the efficiency of the prototype. We shall also use C-Sharp and the .NET framework to come up with the basic prototype and MySQL server engine to store any records electronically in the database if any.

\section{LITERATURE REVIEW}
Medical simulation is a modern day methodology of reenacting real medical situations through the use of advanced educational technology and is used for learning or training in many different medical fields. The main goal of medical simulation is to train students or professionals to reduce the number of mistakes during medical operations. It has always been very difficult to simulate real human organs, however, thanks to technological advances the simulations get more and more realistic over the years. Studies have shown that learning with medical simulations will reduce the chance of accidents significantly.
During a simulation students enter into a realistic medical setting where a high-tech patient mannequin is being operated by medical and technical staff. The patient simulation is recorded for playback during the debriefing process. The educator provides the voice of the patient and also guides the control of the mannequin. He, then, debriefs the students utilizing video playback to recap performance outcomes and to identify errors and successes.


\begin{thebibliography}{10}
\bibitem{latexGuide} The principles of respiration  by Hankiinson ,R.Jenson 
Available at \texttt{ International series of monographs on respiration}
\bibitem{latexGuide}Publisher:Clarendon Press ,2005). 
Available at \texttt{  standardisation of spirometry pages 27, volume 25 ,1986 }
\bibitem{latexGuide}A Modern Approach by C.Busess,P.Pietsch,W.Guggenbuhl and E.A.Koller .
 \emph  A pulsed diagonal beam ultrasonic airflometer  standardisation of spirometry pages 27, volume 25 ,1986
\bibitem{latexGuide} Google
\emph  Dashboards - Android developers, July 2013
Available at \texttt{ http://developer.android.com/about/ dashboards/index.html }
\end{thebibliography}

\end{document}





Contact GitHub API Training Shop Blog About
© 2017 GitHub, Inc. Terms Privacy Security Status Help